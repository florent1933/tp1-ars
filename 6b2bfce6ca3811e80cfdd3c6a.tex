

% !TEX TS-program = xelatex
% !TEX encoding = UTF-8
\documentclass[12pt]{report}
\usepackage{fontspec}
\usepackage{polyglossia}
\setdefaultlanguage{french}
\usepackage{graphicx} % support the \includegraphics command and options
\usepackage{geometry} % See geometry.pdf to learn the layout options. There are lots.
\geometry{a4paper} % or letterpaper (US) or a5paper or....
\usepackage{lmodern}
\usepackage{amssymb,amsmath}


\usepackage{fancyvrb}
\usepackage{longtable}

\usepackage{hyperref}

\usepackage{booktabs}
\usepackage{bookmark}


\hypersetup{breaklinks=true,
            pdfauthor={Arthur JEANNIN\\Florent CLAPIÉ},
            pdftitle={tp-1-igraphr-introductionemphrushed-kanawati},
            colorlinks=true,
            citecolor=blue,
            urlcolor=colourUrl,
            linkcolor=colourLink,
            pdfborder={0 0 0}}
\urlstyle{same}  % don't use monospace font for urls
\setlength{\parindent}{6pt}
\setlength{\parskip}{4pt plus 2pt minus 1pt}
\setlength{\emergencystretch}{3em}  % prevent overfull lines
\setcounter{secnumdepth}{0}
\VerbatimFootnotes % allows verbatim text in footnotes

\usepackage{xcolor}
\definecolor{colourLink}{HTML}{000000}
\definecolor{colourUrl}{HTML}{000000}
\newcommand{\euro}{€}



\title{TP 1 IGRAPH/R - Introduction\\\emph{Rushed KANAWATI}}
\author{Arthur JEANNIN \and Florent CLAPIÉ}
\date{\today}

\begin{document}
\maketitle



\tableofcontents  \newpage  


\section{Introduction}

L'objectif de ce TP est de se familiariser avec R et l'outil igraph : un
outil d'analyse et de visualisation de graphes.

\textbf{}

Ce tp est disponible sur github :
\href{https://github.com/florent1933/tp1-ars}{https://github.com/florent1933/tp1-ars}

\textbf{}

Pour faire fonctionner le code, il suffit de définir le chemin d'accès
vers le dossier du code.

\section{Explication du TP}

\textbf{}

\subsection{Question 1}

\textbf{}

Nous avons écrit une fonction topology qui prend en paramètre un graphe
G. La fonction résume les principales caractéristiques topologiques d'un
graphe :

-- vcount(g) : retourne le nombre de nœuds dans g

-- ecount(g) : retourne le nombre de liens dans g

-- graph.density(g) : donne la densit ́e du graphe g

-- diameter(g) : retourne le diam`etre du graphe g

-- degree(g) : retourne le degrés de chaque nœud dans g

-- degree.distribution(g) : calcule la distribution de degrés de g

--  transitivity(g) : calcule le coefficient de clustering du graphe g
--  average.path.length(g) : retourne la moyenne de tous les plus courts
chemins dans le graphe.

--  shortest.paths(g) : retourne un matrice qui donne les longueurs de
plus courts chemins entre chaque couple de nœuds.

--  betweenness(g) : calcule la centralité d'intermédiarité

--  closeness(g) : calcule la centralité de proximité.

--  is.connected(g) : retourne TRUE si le graphe est connexe.

--  clusters(g) : retourne une lite des composantes connexes dans le
graphe

\subsection{Question 2}

\textbf{}

Pour afficher la distribution de degrés d'un graphe, nous avons écrit
une fonction distribution qui prend en paramètre un graphe G et affiche
la distribution de degrés d'un graphe.

\textbf{}

\begin{center} 

 \includegraphics[width = 0.684285714286\textwidth]{8c4d2b5825c27c405bd7561a0f01d3c643aeb93cc48ffd5016df4998.png}


\end{center}

\begin{itemize}
\item
  Distribution de Karaté
\end{itemize}

\textbf{}

\begin{itemize}
\item
\end{itemize}

\textbf{}

\subsection{Question 3}

\textbf{Comparaison des diamètres entre les différents algos }

\textbf{}

\begin{center} 

 \includegraphics[width = 0.684285714286\textwidth]{54e424bfb04aa449ca359bd37a3a93ada73f966a36da2683bbebb3bc.png}


\end{center}

\textbf{}

\textbf{}

Pour étudier les variations de diamètres et de transitivité en fonction
du nombre de noeuds, nous avons tracé un graphe avec des noeuds qui vont
de 50 à 100 avec un pas de 5 pour chaque algo.

\begin{center} 

 \includegraphics[width = 0.684285714286\textwidth]{27abd17eab777a5f2be24ae34bc51f3ef3951097089a826f7f964c84.png}


\end{center}

\begin{center} 

 \includegraphics[width = 0.684285714286\textwidth]{cd18930fc9788546f3414c86dfd203d68b3e3e82fd535edde70a454c.png}


\end{center}

\begin{center} 

 \includegraphics[width = 0.684285714286\textwidth]{889618549b854eecd98844258724b8e269c2b88d2511f9e01fc794a3.png}


\end{center}

\subsection{Question 4}

Pour la coloration des graphes, nous avons colorié par rapport à la
valeur des noeuds.

Nous avons utilisé un vecteur de couleur contenant le nom de nos
couleurs.

\begin{center} 

 \includegraphics[width = 0.684285714286\textwidth]{77ad8b142505bb3c1ce2cffb81e367f46d632c7fbc1470f90637d1a5.png}


\end{center}

\textbf{Communauté karaté}

\textbf{}

Pour voir les autres communautés, il faut lancer le programme car il y a
trop de noeuds. 

\subsection{Question 5}

Il y a 12 graphes, merci de vous référer au code.

Exemple pour le graphe \textbf{football}

\begin{center} 

 \includegraphics[width = 0.765714285714\textwidth]{42221fa301f11def6c9951ee6be91502590bbaea0df1a28ee458757b.png}


\end{center} \begin{center} 

 \includegraphics[width = 0.765714285714\textwidth]{ea0d0a61fe2c57b31196467b89ccde66edad78c47daba3740aa4daf0.png}


\end{center} \begin{center} 

 \includegraphics[width = 0.765714285714\textwidth]{f68f726570577a236a87fb43048e507a3ec76893c77644c8cca1a3ef.png}


\end{center}

\section{Conclusion}

\textbf{}

Ce TP fut très intéressant, il nous a permis de nous familiariser avec R
bien que nous ayons eu des difficultés.

Nous n'avons pas trouvé comment faire des commentaires multilignes,
concaténer une chaîne + variable de façon correcte :

\begin{verbatim}
cat("nombre de noeuds : ",  vcount(g), "\n")
\end{verbatim}

Il nous a aussi permis d'apprendre à utiliser igraph



\end{document}

